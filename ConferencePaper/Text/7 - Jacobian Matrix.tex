The Jacobian matrices for AFER v1 and v2 were solved in order to provide insight on the relationships between the joint space and Cartesian space when planning the movement patterns. The Jacobian was determined by using standard spherical-Cartesian transformation equations as in (7), and the substitution values for v1 were $\theta = \theta_{1}$, $\phi = \theta_{2}$, and $r = \textit{l2}$.

\[
\centering
\begin{bmatrix}
sin\theta_{2}cos\theta_{1} & \textit{l2}cos\theta_{2}cos\theta_{1} & -\textit{l2}sin\theta_{2}sin\theta_{1} \\
sin\theta_{2}sin\theta_{1} & \textit{l2}cos\theta_{2}sin\theta_{1} & \textit{l2}sin\theta_{2}cos\theta_{1}  \\
cos\theta_{2} & -sin\theta_{2} & 0 \\
\end{bmatrix}
\begin{equation}
\label{eq7}
\end{equation}
\]

The Jacobians for AFER v1 and v2 were identical due to the common core configuration between AFER v1 and v2 and the assumptions regarding J3 in v2. The Jacobian matrix for joints 1 and 2 remained unchanged, and supposing static equilibrium, velocity at the end effector along the y axis will always be zero. The end effector velocity along the x axis could change and, given that the value of $\theta_{3}$ was dependent upon $\theta_{2}$, would always change at the same rate as the velocity at J3. As a result, the Jacobian in (7) sufficed for AFER v2 as well.
