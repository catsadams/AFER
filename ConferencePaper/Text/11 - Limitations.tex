There were considerable limitations to this work including lack of access to a robotics lab due to COVID-19 restrictions; as a result, this robot was developed, built, and tested, entirely with at-home robotics equipment. This severely limited the choice of materials and opportunity to attempt multiple builds. 
Additionally, several necessary base assumptions made limited the scope and real-world applicability portions of the work, particularly for the v2. The assumptions for J3 were significant, yet necessary given the circumstances. As a result, the mathematical model for AFER v2 was not able to account for random motion or vibrations of the string, and, therefore, it was not possible to account for those disturbances while planning movement patterns. Additionally, the lack of an encoder on the passive joint, J3, meant that interpreting the success of the movement patterns in any way other than visual, was not possible.
Finally, the preference analysis conducted was based on only two cats and the testing was not repeated multiple times to ensure consistency. A much larger sample size with repetitive testing events would be needed for robust, statistically significant analysis on animal preferences between movement patterns and end effectors \cite{laschi2006design}.
