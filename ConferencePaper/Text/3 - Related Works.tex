A literature review was conducted in several areas to ensure robust design and experimentation practices.  Previous robotics studies involving pet-robot interactions were investigated to ascertain the level of previous research conducted for similar or analogous robots. Behavioral studies on cats regarding toy and play preferences were researched to inform robot movement patterns and selection of the cat toy to be used as the end effector. 

\subsection{Studies on Pet-Robot Interaction}
Two relevant papers on pet entertainment robots and the related pet-robot interaction were available. The first paper was a design innovation exploring a possible design for an autonomous pet care robot intended for dogs \cite{deng2020design}. The study provided interesting ideas regarding movement and food rewards, but the robot has not yet been prototyped or tested. The second study focused on various types of robots and their effectiveness when entertaining a cat \cite{kim2009animal}. This was more directly applicable to our objectives than the first study, offered good insight into robot design considerations specific to cats, and defined the concept of Animal-Robot Interaction (ARI). 
The model in \cite{deng2020design} is the preliminary design for a “home service robot for pet caring.” This robot design focused on playing, feeding, and otherwise caring for a dog – a set of functions which are not currently available in any single commercial pet-care robots. The research in \cite{kim2009animal} aimed to examine the possibility of a robot replacing a human for the purposes of short-term pet care as well as use the definition and practices of Human-Robot Interaction (HRI) to create a working understanding and set of criteria for Animal-Robot Interaction (ARI). The identified differences between HRI characteristics and ARI characteristics largely hinge on the ability of the user to communicate clearly with the robot. Since a human user has the ability to command the robot to complete a certain task, and an animal user does not, a Pet Care Robot (PCR) must have the ability to accept commands from human users and also have adequate sensing and processing ability to interpret pet-provided signals and respond appropriately to the defined pet behavior. Currently, there are no commercially available robots designed for pet interaction that can monitor and adjust functions based upon the emotional state of the pet.

\subsection{Cat Behavioral Studies}
Many studies on cat play behaviors have been conducted and in reviewing several of those, movement was consistently found to be one of the crucial components when attempting to elicit play behaviors in cats. Movement was found to be particularly effective when the toy was moved away from the cat or in a manner that mimicked prey \cite{rochlitz2005review}, \cite{leyhausen1979cat}. However, examples of effective, real-time and safe interactions between robot toys and cats were limited.