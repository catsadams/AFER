\subsection {Conclusions}
We developed a simple entertainment robot (two versions) to entertain animals (cats) when they are not attended by humans. Despite limited access to materials, the concepts of the robots were shown to be viable, and the kittens/cats were safely engaged and enjoyed playing with the toy actuated by the robot for as long as the robot was operational. The study is new in the robotics field and possesses significant potential to be used enormously to enhance the entertainment, companionship, and well-being of companionship animals.

\subsection {Future Work}
As expected, prototype models of systems had many issues and often required multiple redesigns and repairs; the AFER systems were no different. Unfortunately, redesigns and more robust construction were beyond the scope of the research in its current iteration. The following issues were identified with the design and would need to be addressed in any future work.
     \par The connection at J1 was small and unstable, and a structural redesign of the robot with a more robust joint or additional support material is necessary for any further work. A redesigned system in conjunction with improved modeling procedures would enable refinement of the control algorithm, and lead to better performance. Additionally, implementing more complex movement patterns would increase applicability.
Due to the nature of the end effector attachment on v1, the computer and electrical components of the system needed to be placed on the ground. For a system that is expected to entertain pets, this is not acceptable. In future iterations, implementing control with a localized chip and batteries would be ideal. If this is achieved, the AFER base could be mounted on a mobile robot unit for further autonomous functionality.
    \par The interactions between cats and the robots will need to be videotaped, which will allow further analysis of the behaviors of the cats with the robots. Machine learning methods such as the reinforcement learning, inverse reinforcement learning or deep reinforcement learning will be used to learn the behaviors of the cats with the robots, which may suggest significant improvements in the design and controls of the robots.
