In 2020, pet adoptions steadily rose and the market for autonomous pet entertainment robots is increased concurrently. Pets require mental stimulation and physical exercise to avoid boredom, anxiety, and poor behaviors associated with under stimulation. 

\subsection {Background – Animal Welfare Standards}
‘The Five Freedoms’ were first developed by the Brambell Committee Report in 1965 as the minimum standard for ethical treatment of farm animals. These freedoms are summarized as:
freedom from thirst, hunger, and malnutrition; freedom from discomfort; freedom from pain, injury, and disease; freedom to express normal behavior; freedom from fear and distress \cite{Brambell1965ReportOT}.
   From the Five Freedoms, researchers created a framework for assessing the welfare of zoo and lab animals as well. In the case of companion animals, the responsibility for meeting these needs and providing a healthy and engaging environment as well as the environmental enrichment falls upon their owners. The term ‘environmental enrichment’ can be used to describe “changes, modifications and other interventions made to the environment with the aim of improving the welfare of the animals living in that environment \cite{rochlitz2005review}.”

\subsection {Problem Identification – Providing Adequate Mental Stimulation for Indoor-only Cats}
The American Veterinary Medical Association, Humane Society of the United States, most American animal shelters, and American veterinarians recommend that domestic cats be kept indoors in both urban and suburban living situations. This recommendation considers several factors; outdoor domestic cats can pose a threat to the balance of local ecosystems, and, generally, indoor domestic cats are much healthier and have longer lifespans compared to their outdoor counterparts \cite{rochlitz2005review}. Remaining indoors protects the cats from outdoor hazards such as vehicles and predators, however, it also provides a much lower degree of stimulation and sensory engagement than the outdoors. This lack of engagement can lead to boredom, stress, and ultimately destructive behaviors of the cats \cite{wells2009sensory}. Therefore, indoor cats require enrichment activities, yet awareness of enrichment needs is often lacking among pet owners.
  Some pet owners may find it particularly difficult to meet the sensory and occupational needs of younger cats and kittens, as they tend to have more energy and require more active play time. In order to provide a more enriching environment for these cats, a robot designed to engage and play with the cats could be activated when their owner is otherwise occupied. The robot should provide effective stimulation and engagement, and be safe for the cats. However, there are extremely few reports in the literature investigating effective examples of such entertainment robots for cats or pets. 

\subsection{Objectives}
This project aimed to develop a robot that could fulfill this purpose. Thus, the objective of this paper is to present the design, development, construction, and evaluation of an autonomous toy-like animal robot - the Autonomous Feline Entertainment Robot (AFER) with its two Versions 1(v1) and 2(v2). The robot developed, built, and tested was intended to entertain two six-month old kittens when they were unable to be attended by a human. The AFER system aims to assist pet owners with meeting their needs towards entertaining their pets.