Initial testing of movement patterns on the completely assembled robot was extremely successful, and the robot followed the linear movement pattern as expected. Introducing both kittens to the base configuration and AFER v1 configuration was simple and uneventful. Neither animal exhibited any of the common fear or stress reactions – ears flattened, sitting low, hiding or hissing – and both engaged with the end effector toy once the robot was in motion. 
For the analysis of the pet-robot interaction shown in the tables below, Active Engagement (AE) was defined as the kitten chasing and catching or attempting to catch the end effector; Passive Engagement (PE) was defined as the kitten watching and showing interest in the robot or end effector, but not attempting to chase or catch. No Engagement (NE) was used to describe time in which the kitten was completely disinterested in the device. Engagement behaviors were observed over a 5-minute experimentation period. Comparisons of engagement for AFER v1 is shown in Table 3.

\begin{center}
\begin{table}[htbp]
\centering
\caption{AFER v1 INTERACTION}
    \begin{tabular}{ |c|c|c|c|c|c|c| }
    \hline
    Kitten & AE & \%AE & PE & \%PE & NE & \%NE \\ \hline
    Male & 45s & 15\% & 3min25s & 68\% & 50s & 17\%  \\ \hline
    Female & 3min45s & 75\% & 45s & 15\% & 0s & 0\% \\ \hline
    \end{tabular}
		\label{table:3}
\end{table}
\end{center}

The kittens were also not frightened by the AFER v2 configuration, however, the female kitten seemed more wary and the male kitten dominated the play window. Of the 5 minutes allotted for testing, only 2 minutes were completed prior to the failure of the J1 joint. Time of engagement is shown in Table 4.

\begin{center}
\begin{table}[h]
\centering
\caption{AFER v2 INTERACTION}
    \begin{tabular}{ |c|c|c|c|c|c|c| }
    \hline
    Kitten & AE & \%AE & PE & \%PE & NE & \%NE \\ \hline
    Male & 2min & 100\% & 0s & 0\% & 0s & 0\%  \\ \hline
    Female & 0s & 0\% & 2min & 100\% & 0s & 0\% \\ \hline
    \end{tabular}
		\label{table:4}
\end{table}
\end{center}

The J1 failure occurred when the top portion of the robot including J2 and \textit{l1} separated from the base and J1 at the servomotor-servo horn connection. Despite the superglue used at that connection, the force created by the kittens playing with the end effector was too much for the robot to withstand. Testing of the robot was halted at this point and repairs were attempted. 
Although testing was cut short, both versions of the robot proved viable concepts, and highly interesting to the kittens. Both kittens engaged with both versions of the robot, though each seemed to have a distinct preference in version.
